\section{Предисловие}
\subsection{Туда ли ты зашёл?}
\begin{em}
Одна печатаемая ерунда создает ещё у двух убеждение, что и они могут
написать не хуже. Эти двое, написав и будучи напечатанными, возбуж-
дают зависть уже у четырёх.
\begin{flushright}
В. Маяковский
\end{flushright}
\end{em}
Книга, которую вы держите на своём НЖМД\footnote{Накопитель
на жестких магнитных дисках, он же „винт”}, расскажет вам, как выжить
в городских условиях используя Perl. Хотя книгой сей документ назвать
трудно. Скорее, это конспект или подшивка обучающих статей, обобщение серии
тредов, которые были запилены на известном сайте со смешными картинками.
К чему весь сыр-бор? Я не знаю. Это конечно плохо, когда начинаешь
что-то не имея конкретных целей, но рискнём, ведь так? Могу сказать
лишь, что мы будем не только овладевать искусством программирования. 
Я постараюсь осветить многие стороны околокомпьютерного бытия.
Разумеется речь идёт не об унылом матане, а о чем-то интересном или 
забавном.

Целевой аудиторией считаются гуманитарии и школьники с нулевым
опытом программирования но не нулевой любознательностью. Очень важный
момент! Книга — не кузница кадров для компаний-лидеров в своей
отрасли. Тем более, что я сам не очень-то опытный кодер. Поэтому
учиться будем вместе. А ведь материал усваивается лучше всего тогда,
когда объясняешь другому. Но суть не в этом. Мне очень хочется показать,
что программировать интересно и ученье — свет! Уверен, Антон уже очень
скоро почувствует бабочек в животе от решения интересных проблем.

Почему Perl? Это мощный и гибкий язык, которым легко овладеть, но
стать гуру Perl’a непросто. Хотя это справедливо для любой области
человеческой деятельности. На перле можно писать что угодно. От
„однострочников”, до вакабы и дальше. У Perl’а туго разве что с вещами,
где нужна высокая производительность (крузис на Perl’е не напишешь).
Но это справедливо для всех интерпретируемых языков, таких как python
или php.Восхвалять Perl больше не буду. Кто не с нами — тот под нами!

\subsection{Всей основой!}

Мне кажется, что идея перлотредов/книги очень хороша. К сожалению,
подобные вещи отнимают массу времени и нервов. Поэтому прошу о
снисхождении. Да, я верстаю, как мудак, у меня плохо с правописанием и вообще.
Заранее извиняюсь. Также я прошу о помощи. Принимайте участие, ребята!
Мне интересно услышать ваше мнение о стенах моего текста. Возможно
у тебя, анончик, есть какие-то идеи. Ты знаешь интересный и несложный
алгоритм, задачку, можешь посоветовать полезную статейку, или какой-то скрипт
(из приведенных в книге) стоило реализовать иначе? Буду очень рад,
если художники помогут своими медскилзами. А может просто стоит сказать
автору, что стены унылого текста всех достали.

Кстати, об унынии. Как-то не хочется ругаться в книге, не Интернет все-таки.
Тем не менее, некоторые вещи могут огорчать негров. Если вы один
из них — пишите. Будем принимать меры.

В любом случае, я приглашаю всех принять участие! Давайте ковать
контент всей основой!

\subsection{О нас}

На самом деле, это уже вторая попытка написания данного руковдоства и уверен,
что на этот раз она закончится в Берлине! Теперь я тащу движуху не один, мусьё
Сосницкий, помимо подготовки к ЕГЭ, взвалил на свои плечи значительную часть
нагрузки, а tovarisch ival выказал своё молчаливое согласие.

Ниже расположены наши координаты и ссылка на репозиторий с исходниками книжки.
Пишите письма \mbox{„ОТКРЫТО И СМЕЛО, ПРЯМО В ЛИЦО!”}

\begin{itemize}
\item раки@conference.jabber.ru
\item https://github.com/dasrenat/cancer-vault/
\end{itemize}


